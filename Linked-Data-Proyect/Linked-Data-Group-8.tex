\documentclass[11pt]{report}

% Paquetes y configuraciones adicionales
\usepackage{graphicx}
\usepackage[export]{adjustbox}
\usepackage{caption}
\usepackage{float}
\usepackage{titlesec}
\usepackage{geometry}
\usepackage[hidelinks]{hyperref}
\usepackage{fontspec}

% Configuración de la fuente usada
\setmainfont{Geist}


% Configura los márgenes
\geometry{
    left=2cm,   % Ajusta este valor al margen izquierdo deseado
    right=2cm,  % Ajusta este valor al margen derecho deseado
    top=2cm,
    bottom=2cm,
}

% Configuración de los títulos de las secciones
\titlespacing{\section}{0pt}{\parskip}{\parskip}
\titlespacing{\subsection}{0pt}{\parskip}{\parskip}
\titlespacing{\subsubsection}{0pt}{\parskip}{\parskip}


\begin{document}
	
	% Portada del informe
	
	\title{Linked Data}
	\author{Samuel Martín Morales  \texttt{alu0101359526@ull.edu.es} \and Jorge Domínguez González  \texttt{alu0101330600@ull.edu.es} \and Juan Diego Rendon Cachafeiro \texttt{alu0101327747@ull.edu.es}}
	\date{\today}
	
	\maketitle
	
	% Índice
	\tableofcontents
	
	% Secciones del informe
	\chapter{Introducción}
	
  En la última década, la web ha experimentado una transformación fundamental desde una simple red de información hacia lo que hoy se conoce como \textbf{Linked Data} o \textbf{datos enlazados}. Este cambio, impulsado por la evolución de la web semántica, ha llevado a la adopción de un paradigma que va más allá de la  presentación de información en forma de texto. Linked Data propone una visión donde los datos adquieren una estructura que facilita la creación de conexiones y enlaces entre diversos conjuntos de datos, provenientes incluso de fuentes y proveedores distintos.

De manera general, Linked Data representa un conjunto de prácticas sólidas para la publicación y conexión de datos estructurados en la web. Haciendo uso de tecnologías del W3C, como \textbf{URIs}, el \textbf{protocolo HTTP} y el modelo de datos \textbf{RDF} o \texttt{\textbf{Resource Description Framework}}, se establece una base que permite la identificación única de entidades, la recuperación de recursos y la descripción detallada de los mismos.

En el presente informe se tiene como objetivo explorar los fundamentos de Linked Data, desde sus principios esenciales hasta su aplicación práctica. En esencia este se centrará en cómo las URIs, el protocolo HTTP y el modelo RDF forman parte de la revolución semántica, permitiendo la interconexión de datos. Además, se examinará el impacto del Linked Open Data (LOD) y cómo este enfoque híbrido entre \texttt{datos enlazados} y \texttt{datos abiertos} está transformando la forma en la que se accede, se utiliza y se comparte la información en un mundo cada vez más interconectado.
	 
	\chapter{Componentes del Linked Data}
	Example....
	
	
	\chapter{Propósitos y principios de RDF}
  Example....	

	\chapter{Principios de Linked Data según Tim Berners-Lee}
	Example....

	\chapter{Proyecto Linking Open Data}
	Example....

	\chapter{Linked Open Data (LOD)}
	Example....

	\chapter{Relación entre Linked Data y Open Data}
	Example....

	\chapter{Linked Data y Web Semántica}
	Example....

	\chapter{Linked Data y Web 3.0}
	Example....

	\chapter{Beneficios de Linked Data}
	Example....

	\chapter{Problemas de Linked Data}
	Example....

	\chapter{Linked Data en bibliotecas}
	Example....

	\chapter{Linked Data en la actualidad}
	Example....

	\end{document}