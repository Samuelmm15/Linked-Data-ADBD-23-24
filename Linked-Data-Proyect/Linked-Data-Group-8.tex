\documentclass[11pt]{report}

% Paquetes y configuraciones adicionales
\usepackage{graphicx}
\usepackage[export]{adjustbox}
\usepackage{caption}
\usepackage{float}
\usepackage{titlesec}
\usepackage{geometry}
\usepackage[hidelinks]{hyperref}
\usepackage{fontspec}

% Configuración de la fuente usada
\setmainfont{Geist}


% Configura los márgenes
\geometry{
    left=2cm,   % Ajusta este valor al margen izquierdo deseado
    right=2cm,  % Ajusta este valor al margen derecho deseado
    top=2cm,
    bottom=2cm,
}

% Configuración de los títulos de las secciones
\titlespacing{\section}{0pt}{\parskip}{\parskip}
\titlespacing{\subsection}{0pt}{\parskip}{\parskip}
\titlespacing{\subsubsection}{0pt}{\parskip}{\parskip}


\begin{document}
	
	% Portada del informe
	
	\title{Linked Data}
	\author{Samuel Martín Morales  \texttt{alu0101359526@ull.edu.es} \and Jorge Domínguez González  \texttt{alu0101330600@ull.edu.es} \and Juan Diego Rendon Cachafeiro \texttt{alu0101327747@ull.edu.es}}
	\date{\today}
	
	\maketitle
	
	% Índice
	\tableofcontents
	
	% Secciones del informe
	\chapter{Introducción}
  Example....
	
	\chapter{Componentes del Linked Data}
	Example....
	
	
	\chapter{Propósitos y principios de RDF}
  Example....	

	\chapter{Principios de Linked Data según Tim Berners-Lee}
	Example....

	\chapter{Proyecto Linking Open Data}
	Example....

	\chapter{Linked Open Data (LOD)}
	Example....

	\chapter{Relación entre Linked Data y Open Data}
	Example....

	\chapter{Linked Data y Web Semántica}
	Example....

	\chapter{Linked Data y Web 3.0}
	Example....

	\chapter{Beneficios de Linked Data}
	Example....

	\chapter{Problemas de Linked Data}
	Example....

	\chapter{Linked Data en bibliotecas}
	Example....

	\chapter{Linked Data en la actualidad}
	Example....

	\end{document}