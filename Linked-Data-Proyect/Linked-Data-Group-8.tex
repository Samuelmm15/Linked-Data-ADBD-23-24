\documentclass[11pt]{report}

% Paquetes y configuraciones adicionales
\usepackage{graphicx}
\usepackage[export]{adjustbox}
\usepackage{caption}
\usepackage{float}
\usepackage{titlesec}
\usepackage{geometry}
\usepackage[hidelinks]{hyperref}
\usepackage{parskip}
\usepackage{fontspec}
\usepackage{listings}

% Configuración de la fuente usada
% \setmainfont{Geist}


% Configura los márgenes
\geometry{
    left=2cm,   % Ajusta este valor al margen izquierdo deseado
    right=2cm,  % Ajusta este valor al margen derecho deseado
    top=2cm,
    bottom=2cm,
}

% Redefinir el formato del capítulo
\titleformat{\chapter}[block]
  {\normalfont\huge\bfseries}{\chaptertitlename\ \thechapter}{1em}{\Huge}

% Ajustar el espaciado antes y después del título del capítulo
\titlespacing*{\chapter}{0pt}{0pt}{20pt}

% Configuración de los títulos de las secciones
\titlespacing{\section}{0pt}{\parskip}{\parskip}
\titlespacing{\subsection}{0pt}{\parskip}{\parskip}
\titlespacing{\subsubsection}{0pt}{\parskip}{\parskip}


\begin{document}
	
	% Portada del informe
	
	\title{Linked Data}
	\author{Samuel Martín Morales  \texttt{alu0101359526@ull.edu.es} \and Jorge Domínguez González  \texttt{alu0101330600@ull.edu.es} \and Juan Diego Rendon Cachafeiro \texttt{alu0101327747@ull.edu.es}}
	\date{\today}
	
	\maketitle
	
	% Índice
	\tableofcontents
	
	% Secciones del informe
	\chapter{Introducción}

  En la última década, la web ha experimentado una transformación fundamental desde una simple red de información hacia lo que hoy se conoce como \textbf{Linked Data} o \textbf{datos enlazados}. Este cambio, impulsado por la evolución de la web semántica, ha llevado a la adopción de un paradigma que va más allá de la  presentación de información en forma de texto. Linked Data propone una visión donde los datos adquieren una estructura que facilita la creación de conexiones y enlaces entre diversos conjuntos de datos, provenientes incluso de fuentes y proveedores distintos.

De manera general, Linked Data representa un conjunto de prácticas sólidas para la publicación y conexión de datos estructurados en la web. Haciendo uso de tecnologías del W3C, como \textbf{URIs}, el \textbf{protocolo HTTP} y el modelo de datos \textbf{RDF} o \texttt{\textbf{Resource Description Framework}}, se establece una base que permite la identificación única de entidades, la recuperación de recursos y la descripción detallada de los mismos.

En el presente informe se tiene como objetivo explorar los fundamentos de Linked Data, desde sus principios esenciales hasta su aplicación práctica. En esencia este se centrará en cómo las URIs, el protocolo HTTP y el modelo RDF forman parte de la revolución semántica, permitiendo la interconexión de datos. Además, se examinará el impacto del Linked Open Data (LOD) y cómo este enfoque híbrido entre \texttt{datos enlazados} y \texttt{datos abiertos} está transformando la forma en la que se accede, se utiliza y se comparte la información en un mundo cada vez más interconectado.
	 
	\chapter{Componentes del Linked Data}

	Para comenzar con el estudio de Linked Data, es necesario entender los componentes que lo conforman. En este sentido, se puede decir que Linked Data se basa en tres pilares fundamentales: URIs, HTTP y RDF. A continuación, se describirá cada uno de estos componentes y se explicará su importancia en el contexto de Linked Data.

- URI (Identificadores de recursos uniformes): Una URI es una cadena de caracteres que identifica de manera única un recurso en la web. En este perspectiva, se puede decir que una URI es un identificador de recursos uniforme, ya que, permite la identificación de los distintos recursos en la web de una manera uniforme y consistente. Además, las URIs son utilizadas por los agentes de software para acceder a los recursos de esta. La uniformidad en el contexto de las URIs hace referencia a los siguiente aspectos:

\indent \indent \indent -  \textbf{Unicidad}: Cada recurso debe tener una URI única. La unicidad garantiza que no haya conflictos ni duplicados en la identificación de los recursos. Cada URI debería de ser única en el ámbito global de la web.

\indent \indent \indent -  \textbf{Consistencia}: Las URIS deben de seguir un formato consistente y estandarizado. Esto permite la facilidad de comprensión y manejo por parte tanto de las máquinas como de las personas. Además, permite el establecimiento de patrones y la simplificación de su uso.

\indent \indent \indent -  \textbf{Persistencia}: Las URIs deben de ser persistentes. Esto quiere decir que una URI debe de ser válida y accesible en todo momento. De esta manera, se garantiza que los recursos puedan ser accedidos en el momento que se considere.

\indent \indent \indent -  \textbf{Desreferenciable}: Las URIs deben de ser desreferenciables. Es decir, el acceder a una URI mediante el protocolo \texttt{HTTP} se debe de obtener información sobre el recurso al que hace referencia la URI. Esto permite que las URIs no solo se traten de identificadores únicos, sino 	que también sean enlaces a información relevante.

- HTTP (Protocolo de Transferencia de Hipertexto): Se hace uso del protocolo HTTP para que las URIs sean desreferenciables. Esto quiere decir que, al acceder a una URI mediante el protocolo HTTP se debe de obtener información sobre el recurso al que hace referencia la URI. Además, el protocolo HTTP permite la recuperación de recursos a través de la web. En este sentido, se puede decir que el protocolo HTTP es el protocolo de la web, ya que, es el protocolo que permite la recuperación de recursos a través de esta. En cuanto a las características del protocolo, se pueden encontrar las siguientes a continuación:

\indent \indent \indent - \textbf{Cliente-Servidor}: El protocolo HTTP se basa en un modelo cliente-servidor. Esto quiere decir que, el cliente realiza una petición al servidor y este le responde con la información solicitada. En este sentido, el cliente es el agente de software que realiza la petición y el servidor es el agente de software que responde a la petición.

\indent \indent \indent - \textbf{Sin estado}: El protocolo HTTP es un protocolo sin estado. Es decir, cada petición que se realiza al servidor es independiente de las demás. Por tanto, el servidor no guarda información sobre las peticiones anteriores. Esto permite que el protocolo sea simple y fácil de implementar.

\indent \indent \indent - \textbf{Protocolo de aplicación}: Se trata de un protocolo de nivel de aplicación utilizado para la transferencia de información en la WWW (\texttt{World Wide Web}). De manera general, opera en la capa de aplicación del modelo OSI (\texttt{Open Systems Interconnection}) \cite{1}. 

\indent \indent \indent - \textbf{Mensajes}: Las comunicaciones se realizan mediante mensajes. Un solicitud de un cliente y una respuesta del servidor consisten en un encabezado y de manera opcional en un cuerpo. El encabezado contiene la información sobre dicha solicitud o respuesta y el cuerpo contiene la infomación que se quiere transmitir.

\indent \indent \indent - \textbf{URis}: Las distintas solicitudes y respuestas en HTTP hacen uso de identificadores de recursos uniformes para identificar los distintos recursos en la web. Esto lo que permite, es especificar la ubicación y el nombre del recurso.

\indent \indent \indent - \textbf{Basado en texto}: Las distintas solicitudes y respuestas se codifican en texto.  Esto lo que permite es que se facilita la comprensión y el procesamiento de las solicitudes y respuestas tanto por parte de los humanos como de las propias máquinas.

\indent \indent \indent - \textbf{Métodos}: El protocolo define una serie de métodos que se utilizan para poder indicar la acción que el cliente quiere realizar. Algunos de los métodos más comunes son GET, POST, PUT, DELETE, etc.

- RDF (Marco de Descripción de Recursos): RDF es un modelo estándar para describir recursos y sus relaciones haciendo uso de tripletes (sujeto, predicado, objeto).

\indent \indent \indent - \textbf{Reutilización}: RDF hace uso de URIs para identificar los recursos, permitiendo la facilidad en la reutilización de la información RDF en diferentes aplicaciones.

\indent \indent \indent - \textbf{Interoperabilidad}: Este está estandarizado por el W3C \cite{5}, lo que permite que diferentes aplicaciones RDF puedan trabajar juntas más fácilmente.

\indent \indent \indent - \textbf{Extensibilidad}: RDF permite la extensión de vocabularios RDF existentes, lo que permite la creación de vocabularios RDF más específicos.

\indent \indent \indent - \textbf{Escalabilidad}: Se puede hacer más grande según sea necesario (escalable),  permitiendo que se pueda usar para representar grandes cantidades de información o datos.

- Enlaces entre recursos: Las URIs deben de incluir enlances (enlaces de hipertexto) a otras URIs, de esta manera se pueden establecer relaciones entre los recursos y la navegación entre estos. Esto lo que permite es la fomentación de la creación de una red interconectada de datos en la web.

Estos componentes permiten que los datos estén interconectados, facilitando por un lado la navegación y el descubrimiento de información relacionada. Por tanto, cuando se siguen estos principios, se puede decir que los datos están enlazados (\textbf{Linked Data}), siendo estos datos fundamentales para la construcción de la web semántica, dónde, la información tiene un significado definido y las máquinas pueden entender y procesar los datos de manera efectiva.

Teniendo en cuenta todo esto anterior, se puede observar a continuación una esquema de todo lo comentado anteriormente, de tal manera que los conceptos básicos puedan ser comprendidos de manera más sencilla \ref{fig:Componentes-Linked-Data}.

\begin{figure}[H]
	\centering
	\includegraphics[scale=0.6]{../img/Componentes-Linked-Data.png}
	\caption{Esquema de los principales componentes de Linked Data..}
	\label{fig:Componentes-Linked-Data}
\end{figure}

	\chapter{RDF de manera básica}
  Como se ha comentado de manera previa, \texttt{RDF} es un \textbf{modelo} que permite representar propiedades y valores de propiedades. Este, se basa en principios que se encuentran establecidos haciendo uso de varios tipos de representación de datos. Por otro lado, se puede hablar de las \textbf{propiedades RDF}, estas, se asemejan a los atributos y de manera general se corresponden con los pares de atributo-valor. Por último, se puede hablar de los \textbf{recursos RDF}, estos, se asemejan a los objetos en cuanto a aspectos de la terminología del diseño orientado a objetos, por tanto, los \texttt{recursos} son los objectos y las \texttt{propiedades} son los objetos específicos y variables de una categoría.

	En cuanto al modelo de datos básico de \textbf{RDF} se basa en los siguientes tres tipos de objetos:

	- \textbf{Recursos}: Los recursos se tratan de los objetos que se quieren describir, es decir, se enfocan como todos los elementos que son descritos por expresiones RDF. Es decir, por ejemplo, un recurso puede se una página web completa, una parte de una página web, una colección completa de páginas, etc. Para finalizar, los recursos se deben de designar siempre por URIs más etiquetas de identificación de destino.

	- \textbf{Propiedades}: Las propiedades se tratan de los atributos que se quieren describir, es decir, se trata de un aspecto específico, característica, atributo o relación utilizado para describir un recurso. Por ejemplo, una propiedad puede ser el título de una página web, el autor de una página web, la fecha de creación de una página web, etc.  Cada propiedad tiene un significado en específico, define los valores permitidos, los tipos de recursos que puede describir y las relaciones con otras propiedades.

	- \textbf{Setencias}: Las sentencias se tratan de una expresión que relaciona un recurso con una propiedad y un valor. Las tres partes individuales de una sentencia se denominan como \textbf{sujeto (\texttt{recurso}), predicado (\texttt{propiedad}) y objeto (\texttt{valor de la propiedad o literal})}, es decir, los denominados \textbf{tripletes}. El objeto de una sentencia puede ser otro recurso o un valor literal, a su vez un recurso puede ser especificado por un URI o una cadena simple de caracteres que se denominan como \textbf{literales}, además, dicho recurso puede ser a su vez datos primitivos definidos por \textbf{XML} (Lenguaje de marcado extensible) \cite{6}. 

Para poder comprender todo esto anterior, se tiene el siguiente ejemplo de setencia RDF:

\begin{verbatim}
	http://www.example.es/index.html tiene una desarrolladora cuyo valor es Andrea López.
\end{verbatim}

\begin{quote}
	\textbf{Sujeto}: http://www.example.es/index.html

	\textbf{Predicado}: "desarrollador"

	\textbf{Objeto}: Andrea López 
\end{quote}	

	Tras esto, se puede observar el diagrama de nodo y arco simple que representa el ejemplo adjunto anteriormente:

	\begin{figure}[H]
		\centering
		\includegraphics[scale=0.7]{../img/Diagrama-Nodo-Arco.png}
		\caption{Diagrama de nodo y arco simple que representa el ejemplo de setencia RDF.}
		\label{fig:Diagrama-Nodo-Arco}
	\end{figure}

	\fbox{\parbox{\textwidth}{
    \textbf{Nota:} La dirección de la flecja del arco simple es muy importante. El arco siempre debe de empezar en el sujeto y apunta hacia el objeto de la sentencia RDF. De manera general se puede usar la siguiente sentencia:

		\textbf{<Sujeto> TIENE <Predicado> <Objeto>}
}}

A partir del ejemplo comprendido de manera previa, se pueden especificar algunas características más para dicho ejemplo como:

\begin{verbatim}
	El indiviudo cuyo nombre es Andrea López, correo electrónico <andre@example.es>, 
	es la desarrolladores de la página web <http://www.example.es/index.html>.
\end{verbatim}

Teniendo esto en cuenta, la intención del ejemplo se basa en darle valor a la propiedad \textbf{desarrolladora} mediante una entidad estructurada. Por tanto, se puede observar a continuación el diagrama de nodo y arco simple que representa el ejemplo adjunto anteriormente:

\begin{figure}[H]
	\centering
	\includegraphics[scale=0.7]{../img/Propiedad-Estructurada.png}
	\caption{Diagrama de nodo y arco simple que representa el ejemplo de setencia RDF.}
	\label{fig:Diagrama-Nodo-Arco-2}
\end{figure}

\fbox{\parbox{\textwidth}{
	\textbf{Nota:} Se puede leer el diagrama anterior como http://www.example.es/index.html tiene la desarrolladora cualquiera y este cualquiera tiene el nombre Andrea López y el correo electrónico andre@example.es .
}}

Para finalizar, se puede implementar como ejemplo el uso de múltiples frases o sentecias RDF, por tanto, se tiene el siguiente ejemplo para ello:

\begin{verbatim}
	El individuo al que se refiere el identificador de empleado id 15151 se llama Andrea López y 
	tiene la dirección de correo electrónico <andre@example.es>. Este individuo creó el 
	recurso <http://www.example.es/index.html> y es la desarrolladora de este recurso.
\end{verbatim}

Con esto en mente, se implementad el siguiente \textbf{modelo RDF}:

\begin{figure}[H]
	\centering
	\includegraphics[scale=0.7]{../img/Modelo-RDF.png}
	\caption{Modelo RDF que representa el ejemplo de múltiples sentencias RDF.}
	\label{fig:Modelo-RDF}
\end{figure}

	\chapter{RDF y XML}
	SAMUEL
	\url{https://arxiu-web.upf.edu/hipertextnet/numero-1/rdf.html#2.2}

	\chapter{Principios de Linked Data según Tim Berners-Lee}
	SAMUEL

	\chapter{Linked Open Data (LOD)}
		Linked Open Data (LOD), o Datos Vinculados Abiertos en español, representa un conjunto de prácticas fundamentales para la publicación y conexión de datos estructurados en la World Wide Web. La premisa central de LOD es hacer que los datos estén disponibles de una manera que facilite su acceso, búsqueda y reutilización, fomentando así la interconexión de conjuntos de datos diversos. Estas prácticas se respaldan en estándares internacionales establecidos por el World Wide Web Consortium (W3C).

		\section{Orígenes y Motivación de LOD}

		LOD encuentra sus raíces en la necesidad de superar los desafíos asociados con la diversidad de formatos de datos en la web, que incluyen archivos como PDF, TIFF, CSV y documentos de Word. Aunque estos datos son accesibles a través de enlaces HTML, su procesamiento automatizado a menudo requiere esfuerzos adicionales debido a la falta de estructura semántica.

		\section{Principios Clave de LOD}

		La implementación de LOD busca establecer una forma universal para que cualquier persona pueda leer, compartir y reutilizar datos en la web. La clave para esto es la interrelación de datos, donde diferentes conjuntos de datos están enlazados entre sí, permitiendo una mayor eficiencia en la búsqueda y utilización de información.

		\section{Beneficios de LOD}

		La adopción de LOD conlleva varios beneficios, incluyendo:

		\begin{itemize}
		\item \textbf{Accesibilidad Mejorada:} Al seguir estándares y principios de LOD, se mejora la accesibilidad de los datos, permitiendo su fácil recuperación y uso.
		
		\item \textbf{Eficiencia en la Búsqueda:} La interconexión de datos en LOD facilita la búsqueda y recuperación de información de manera más eficiente.
		
		\item \textbf{Reutilización de Datos:} LOD fomenta la reutilización de datos al establecer un marco que facilita compartir y combinar conjuntos de datos de manera significativa.
		
		\item \textbf{Aplicaciones Interconectadas:} La interrelación de datos en LOD facilita la creación de aplicaciones interconectadas que pueden aprovechar la diversidad de información disponible en la web.
		\end{itemize}

		\begin{figure}[H]
			\centering
			\includegraphics[scale=0.2]{../img/What-are-Linked-Data-and-Linked-Open-Data.png}
			\caption{Linked Data y Linked Open Data.}
			\label{fig:LOD}
		\end{figure}


		\chapter{Relación entre Linked Data y Open Data}

		La relación entre Linked Data y Open Data es esencial para comprender cómo la combinación de estos enfoques puede potenciar la gestión y la utilidad de los conjuntos de datos. A continuación, se profundizará en algunos aspectos clave de esta relación:
		
		\section{Interoperabilidad y Enlace de Datos}
		
		La interoperabilidad destaca como uno de los beneficios más notables de combinar Linked Data y Open Data. Linked Data se centra en la creación de enlaces entre recursos mediante identificadores únicos (URI), facilitando la conexión de conjuntos de datos diversos. Cuando los conjuntos de datos se publican como Open Data y siguen los principios de Linked Data, se establecen puntos de conexión claros y consistentes. Esto permite a diferentes sistemas y aplicaciones interactuar y compartir información de manera más efectiva, promoviendo un ecosistema de datos más amplio y conectado.
		
		\section{Descubrimiento y Explotación de Datos}
		
		La combinación de Linked Data y Open Data abre oportunidades para el descubrimiento y la explotación innovadora de datos. Al utilizar identificadores únicos y estándares semánticos, es posible descubrir relaciones y patrones complejos que podrían pasar desapercibidos en un enfoque más tradicional. Esto es especialmente valioso en contextos científicos, empresariales y gubernamentales, donde la capacidad de realizar descubrimientos significativos puede tener un impacto notable en la toma de decisiones y el avance del conocimiento.
		
		\section{Transparencia y Confianza en los Datos}
		
		La transparencia, un principio clave de Open Data, se ve reforzada por la aplicación de Linked Data al proporcionar una estructura semántica clara. Cuando los conjuntos de datos abiertos están enlazados de manera efectiva, se pueden seguir las relaciones entre diferentes entidades, aumentando así la transparencia y la comprensión de cómo se recopilaron y relacionaron los datos. Esta transparencia fortalece la confianza en los datos, un factor crítico en entornos gubernamentales, científicos y empresariales.
		
		\section{Potencial para la Innovación y la Creación de Valor}
		
		La combinación de Linked Data y Open Data crea un entorno propicio para la innovación y la creación de valor. Al permitir que desarrolladores, investigadores y profesionales accedan y vinculen datos de manera más eficiente, se generan nuevas oportunidades para el desarrollo de aplicaciones, la investigación interdisciplinaria y la creación de servicios que aprovechan la riqueza de información disponible. Esta interconexión de datos puede ser un motor clave para la innovación en áreas como la inteligencia artificial, la ciencia de datos y la toma de decisiones basada en datos.
		
		\section{Desafíos y Consideraciones Éticas}
		
		Aunque la convergencia de Linked Data y Open Data ofrece beneficios significativos, también plantea desafíos. La privacidad y la seguridad son consideraciones críticas, especialmente en el caso de datos abiertos enlazados. Es esencial implementar medidas de seguridad robustas y abordar las preocupaciones éticas para garantizar que la apertura y la interconexión de datos no comprometan la privacidad de las personas ni generen riesgos indebidos.
	\chapter{Linked Data y Web Semántica}
	En el continuo desarrollo de la web, dos conceptos fundamentales han surgido como pilares para la construcción de una web más inteligente y significativa: Linked Data y Web Semántica. Este informe explora la relación intrínseca entre estos dos conceptos, resaltando sus roles individuales y cómo, en conjunto, contribuyen a una web más conectada y rica en significado.
	
	\section*{Linked Data: Interconexión en la Web}
	
	Linked Data, propuesto por Tim Berners-Lee, es un enfoque que busca extender la web actual mediante la creación de conexiones significativas entre datos dispersos. Los principios clave de Linked Data incluyen:
	
	\begin{enumerate}
		\item \textbf{URIs Únicos:} Cada recurso tiene un identificador único (URI) para facilitar su identificación.
	   
		\item \textbf{Uso de RDF:} Descripción de recursos mediante triples RDF (sujeto, predicado, objeto) para agregar contexto semántico.
		
		\item \textbf{Enlaces entre Recursos:} Creación de enlaces entre datos para construir una red interconectada.
	\end{enumerate}
	
	Linked Data proporciona una base sólida para la creación de una web donde los datos están interrelacionados, permitiendo un descubrimiento más eficiente y una comprensión más profunda.
	
	\section*{Web Semántica: Dando Significado a los Datos}
	
	La Web Semántica es una extensión de la web actual que busca agregar un nivel de significado a la información, permitiendo a las máquinas comprender y procesar datos de manera más inteligente. Algunas características clave de la Web Semántica son:
	
	\begin{enumerate}
		\item \textbf{Lenguajes Semánticos:} Uso de estándares como RDF y OWL para representar información de manera semántica.
	   
		\item \textbf{Interconexión de Datos:} Enlazar datos de manera que las relaciones sean comprensibles tanto para humanos como para máquinas.
		
		\item \textbf{Agentes Inteligentes:} Máquinas capaces de razonar sobre datos y realizar tareas más complejas.
	\end{enumerate}
		
	\begin{figure}[H]
		\centering
		\includegraphics[scale=0.4]{../img/websemantica.jpg}
		\caption{Relación entre Linked Data y Web Semántica.}
		\label{fig:websemantica}
	\end{figure}

	\section{Enriquecimiento Semántico de Datos}

	La \textbf{Web Semántica} busca enriquecer los datos al agregar capas de significado semántico. Esto se logra mediante el uso de estándares como \textbf{RDF} (Resource Description Framework) y \textbf{OWL} (Web Ontology Language), que permiten representar relaciones y significados de manera formal y estructurada.

	\section{Interoperabilidad entre Aplicaciones}

	Un objetivo clave de la \textbf{Web Semántica} es mejorar la interoperabilidad entre aplicaciones y sistemas. Al utilizar estándares semánticos, los datos se vuelven más comprensibles para las máquinas, lo que facilita la integración y el intercambio de información entre diferentes plataformas.

	\section{Descubrimiento de Conocimiento}

	La \textbf{Web Semántica} permite un mejor descubrimiento de conocimiento al facilitar la identificación de patrones y relaciones en los datos. Al agregar semántica a los datos, se mejora la capacidad de las máquinas para inferir información y descubrir conocimiento implícito.

	\section{Agentes Inteligentes y Automatización}

	La presencia de agentes inteligentes, impulsados por la semántica de los datos, permite la automatización de tareas más complejas. Estos agentes pueden razonar sobre la información, realizar consultas avanzadas y ejecutar acciones basadas en reglas semánticas predefinidas.

	\section{Búsqueda Semántica}

	La \textbf{Web Semántica} impulsa el desarrollo de motores de búsqueda semántica, que van más allá de la coincidencia de palabras clave y comprenden el significado detrás de las consultas. Esto mejora la precisión de los resultados de búsqueda al considerar el contexto semántico de la información.

	\section{Evolución hacia la Web 3.0}

	A menudo, se asocia la \textbf{Web Semántica} con la transición hacia la \textbf{Web 3.0}, que representa la próxima fase de la evolución de la web. La \textbf{Web 3.0} se caracteriza por una web más inteligente, descentralizada y orientada a la semántica, donde las máquinas pueden comprender y utilizar el contenido de manera más eficiente.

	\section*{Integración de Linked Data y Web Semántica}
	
	La relación entre Linked Data y Web Semántica es de complementariedad. Linked Data establece la estructura y la interconexión, mientras que la Web Semántica agrega capas de significado para una comprensión más profunda. Puntos clave de integración incluyen:
	
	\begin{enumerate}
		\item \textbf{Contextualización Semántica:} La Web Semántica aporta capas de significado a los datos vinculados, mejorando su comprensión y utilidad.
		
		\item \textbf{Agentes Inteligentes en Datos Conectados:} La presencia de agentes inteligentes se beneficia enormemente de la estructura y la interconexión proporcionadas por Linked Data.
		
		\item \textbf{Interoperabilidad Mejorada:} La combinación de Linked Data y Web Semántica impulsa la interoperabilidad, facilitando el uso conjunto de datos heterogéneos.
	\end{enumerate}
	
	\section*{Desafíos y Consideraciones}
	
	Aunque la integración de Linked Data y Web Semántica ofrece beneficios sustanciales, también presenta desafíos. La estandarización de ontologías, la privacidad y la seguridad son consideraciones críticas que deben abordarse para garantizar un desarrollo ético y sostenible.
	
	\chapter{URI- identificadores de recursos uniformes}
	El URI, que significa Identificador Uniforme de Recursos, desempeña un papel crucial en la identificación de recursos en Internet. Su formato estándar proporciona una estructura consistente para identificar una variedad de recursos, como páginas web, servicios, imágenes y vídeos, facilitando la interacción entre ellos. Es esencial destacar que, aunque comúnmente se confunden con los URL, los URI incluyen a los URL. La distinción clave radica en que los URI se centran en la identificación, mientras que los URL se enfocan en la localización.
	\begin{figure}[H]
		\centering
		\includegraphics[scale=0.3]{../img/uri.png}
		\caption{URI vs URL.}
		\label{fig:uri}
	\end{figure}

	El URI se compone de una secuencia de caracteres que describe el método de acceso, la ubicación del recurso y su nombre. Para ilustrar, si consideramos el nombre propio "Luis Castro" como una identificación (similar a un URI), no proporciona información sobre cómo localizarnos. En cambio, una dirección específica se asemejaría a un URL, ya que indica exactamente cómo encontrarnos y establecer contacto.
	
	\section*{Uniform Resource Name (URN)}
	
	El URN, o Nombre Uniforme de Recurso, es una variante del URI que se centra en la identificación mediante el uso de nombres en lugar de localizadores. Aunque los nombres pueden ser menos amigables que los propios, su objetivo es garantizar la identificación única de un recurso.
	
	\subsection*{Características Clave de los URIs:}
	
	\begin{itemize}
		\item \textbf{Uniformidad:} Permite la utilización de diferentes tipos de identificadores en un mismo contexto, incluso cuando los mecanismos de acceso difieren.
		\item \textbf{Recursos:} El término "recurso" abarca una amplia gama de entidades, desde documentos electrónicos hasta servicios y conceptos abstractos.
		\item \textbf{Identificador:} Incorpora la información necesaria para distinguir un recurso de otros dentro de su ámbito de identificación.
		\item \textbf{Ejemplo Ilustrativo:} Se compara el URI con un nombre y el URL con una dirección, subrayando la diferencia entre identificación y localización.
	\end{itemize}
	
	\section*{Ontología}
	
	La ontología, según la definición de Gruber, constituye una especificación formal y explícita de una conceptualización compartida. En el contexto informático, se presenta como una jerarquía de conceptos con atributos y relaciones, proporcionando un vocabulario para describir un dominio específico. Este enfoque es particularmente valioso en el ámbito del Linked Open Data (LOD), donde las ontologías facilitan la interoperabilidad semántica entre sistemas de información al definir reglas, axiomas e instancias. Asimismo, permiten inferir conocimiento a partir de las relaciones y restricciones entre conceptos, contribuyendo a un mejor entendimiento y compartición de información en diversas áreas de conocimiento.
	
	\chapter{Linked Data y Web 3.0}
	Example....

	\chapter{Beneficios de Linked Data}
	Example....

	\chapter{Problemas de Linked Data}
	Example....

	\chapter{Linked Data en bibliotecas}
	Example....

	\chapter{Linked Data en la actualidad}
	Example....

	\chapter{Conclusiones}
	Example....

	\begin{thebibliography}{99}
	\bibitem{1} CloudFlare. 2023.  ¿Qué es el modelo OSI?.  CloudFlare.  \url{https://www.cloudflare.com/es-es/learning/ddos/glossary/open-systems-interconnection-model-osi/}
	\bibitem{2} 12Características. 2023. HTTP (Características, concepto y funciones). 12Características. \url{https://www.12caracteristicas.com/http/}
	\bibitem{3} REDTECA. 2023. Protocolo HTTP. REDTECA. \url{https://redteca.com/hostings/http}
	\bibitem{4}OpenData Euskadi. 2023. RDF (Resource Description Framework). OpenData. \url{https://opendata.euskadi.eus/contenidos/informacion/opendata_rdf_euskadi/es_info/adjuntos/RDF.pdf}
	\bibitem{5} ARITMETICS. 2023. Qué es W3C. ARITMETICS. \url{https://www.arimetrics.com/glosario-digital/w3c}
	\bibitem{6} AWS. 2023. ¿Qué es XML?. Amazon. \url{https://aws.amazon.com/es/what-is/xml/}
	\bibitem{7}.  Eva Méndez. 2001. Resource Description Framework(RDF)	Especificación del Modelo y la Sintaxis. SIDAR. \url{http://www.sidar.org/recur/desdi/traduc/es/rdf/rdfesp.htm#basic}
	\bibitem{8} María Jesús Lamarca Lapuente. 2018. Hipertexto, el nuevo concepto de documento en la cultura de la imagen. Hipertexto. \url{http://www.hipertexto.info/documentos/rdf.htm}
	\end{thebibliography}

	\end{document}